
% Thesis Abstract -----------------------------------------------------


%\begin{abstractslong}    %uncommenting this line, gives a different abstract heading
\begin{abstracts} 
Los sistemas de comunicación ópticos representan en la actualidad una de las tecnologías más empleadas en el sector de las telecomunicaciones, dada su eficiencia, velocidad y demás ventajas provistas por éste desarrollo. Desde una visión esencialista, un sistema de comunicación óptico se compone por un transmisor, un medio de transmisión (Principalmente fibra óptica) y un receptor.

La adecuación de señales eléctricas al medio óptico es en términos generales, la principal tarea ejecutada por un transmisor siguiendo el paradigma actual; además, éste principio no limita sus aplicaciones a un sistema de telecomunicaciones, sino también a sistemas de medición, reflectómetros y métodos de monitoreo y sensado.

El presente trabajo se ocupa del diseño, análisis, desarrollo e implementación de un sistema transmisor óptico, desde la generación de una señal eléctrica hasta su adecuación al medio óptico.   

\end{abstracts}
%\end{abstractlongs}


% ----------------------------------------------------------------------