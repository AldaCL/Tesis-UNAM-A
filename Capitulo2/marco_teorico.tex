
%%%%%%%%%%%%%%%%%%%%%%%%%%%%%%%%%%%%%%%%%%%%%%%%%%%%%%%%%%%%%%%%%%%%%%%%%
%           Capítulo 2: MARCO TEÓRICO - REVISIÓN DE LITERATURA
%%%%%%%%%%%%%%%%%%%%%%%%%%%%%%%%%%%%%%%%%%%%%%%%%%%%%%%%%%%%%%%%%%%%%%%%%

\chapter{Marco teórico}

En éste capítulo se expone el respaldo teórico que sustenta el diseño e implementación de un transmisor láser pulsado.

De forma inicial se describen los elementos constitutivos de un sistema de comunicación óptica, enfocado en el papel del transmisor y sus componentes; posteriormente se describen las características de los microcontroladores de la familia SMT32, la arquitectura arm, el entorno de desarrollo y la estructura del código necesario para su operación. A continuación, se explica el funcionamiento de un amplificador operacional como una fuente de corriente controlada, su respuesta en frecuencia y se revisa la bibliografía existente específicamente en materia de drivers de corriente para diodos láser; finalmente se presenta el principio de operación del diodo láser como una fuente de radiación coherente así como los esquemas de modulación existentes.
%%%%%%%%%%%%%%%%        Subchapter 1     %%%%%%%%%%%%%%%%%%%%%%%%%%%%%%
\section{Sistemas de comunicación óptica}

Un sistema de comunicación óptica, como el caso partícular de uns sistema de comunicación 


%%%%%%%%%%%%%%%%        Subchapter 2     %%%%%%%%%%%%%%%%%%%%%%%%%%%%%%
\section{Familia de microcontroladores STM32}


%%%%%%%%%%%%%%%%        Subchapter 3     %%%%%%%%%%%%%%%%%%%%%%%%%%%%%%
\section{Fuentes de corriente controlada}



%%%%%%%%%%%%%%%%        Subchapter 4     %%%%%%%%%%%%%%%%%%%%%%%%%%%%%%
\section{Principio de operación del diodo láser}
